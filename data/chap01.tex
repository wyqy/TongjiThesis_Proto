\chapter{中文引言~English Introduction~中文引言}
\label{chap:intro}

\section{文字排版示例}
\label{sec:intro:typesetting}
% 文字
大家好啊,我是一段文字。

% 另一段文字
标点挤压测试:我是另一段文字。中文~English~中文,中文(English,En)中文。

汉字排版测试:
我是一段\CJKunderdot{加点和\CJKunderline{下划线的汉字}}。
我是一段\CJKunderline{加下划线和\CJKunderdot{点的汉字}}。

横排引号测试:从前有座山,山里有座庙,庙里有两个和尚,小和尚让老和尚讲故事。老和尚说:
“从前有座山,山里有座庙,庙里有两个和尚,小和尚让老和尚讲故事。老和尚说:
‘从前有座山,山里有座庙,庙里有两个和尚……’”

For users who wants to use Roman Quotation Marks in English-only text:
Please use \verb|\tjeng| environment.
See \href{https://github.com/CTeX-org/ctex-kit/issues/389}{ctex-issue-281} for more information.
For example:

\begin{tjeng}
    Nested Quotation Test: A long time ago in a mountain, there was a temple where two monks lived.
    The young monk asked the old monk to tell a story, and the old monk said:
    “A long time ago in a mountain, there was a temple where two monks lived.
    The young monk asked the old monk to tell a story, and the old monk said:
    ‘A long time ago in a mountain, there was a temple where two monks lived...’”
\end{tjeng}

Hyphenation Test: -- A --, -- 汉字 --, --- A ---, --- 汉字 ---

Ligature Test: significance, difference, stationary (\textit{Times New Roman} does not support it)

我是普通样式的文字。This is a pain-styled sentence.

\textbf{我是加粗样式的文字。This is a bold-styled sentence.}

\textit{我是斜体样式的文字。This is a italic-styled sentence.}

我是带有脚注的文字 \footnote{我是脚注。}。

\section{公式排版示例}
\label{sec:intro:formula}
% 公式
我是一段带有行内公式~$E = mc^2$ 的文字。
我是一段带有如下行间公式~\ref{eq:1} 的文字:
\begin{equation} \label{eq:1} \equcaption{线性回归}
    \vec{y} = \mat{\Psi} \vec{\theta} + \vec{e}.
\end{equation}
行间公式~\ref{eq:1} 已经被列入~\textbf{\listequationname}~之中。
我是一段带有标量~$a$,向量~$\vec{x}$ 和矩阵~$\mat{A}$ 的文字。
下面展示了一组具有子公式编号(如:公式~\ref{eq:2:d})的公式:
\begin{subequations} \begin{align}
    \vec{y}\sps[t] &= \begin{bmatrix*} y\sps[n](t_1) & y\sps[n](t_2) & \cdots & y\sps[n](t_N) \end{bmatrix*}, \\
    \vec{\theta}\sps[t] &= \begin{bmatrix*} \alpha_1 & \cdots & \alpha_{r_a} & \beta_0 & \cdots & \beta_{r_b} \end{bmatrix*}\tr, \\
    \vec{e}\sps[t] &= A\sps[c](p) \begin{bmatrix*} \varepsilon(t_1) & & \varepsilon(t_2) & \cdots & \varepsilon(t_N) \end{bmatrix*}\tr, \\
    \mat{\Psi}\sps[nt] &= \left[\begin{array}{ ccc | ccc }
        -y\sps[n(1)](t_1) & \cdots & -y\sps[n(r_a)](t_1) & u(t_1) & \cdots & u\sps[(r_b)](t_1) \\
        -y\sps[n(1)](t_2) & \cdots & -y\sps[n(r_a)](t_2) & u(t_2) & \cdots & u\sps[(r_b)](t_2) \\
        \vdots & \ddots & \vdots & \vdots & \ddots & \vdots \\
        -y\sps[n(1)](t_N) & \cdots & -y\sps[n(r_a)](t_N) & u(t_N) & \cdots & u\sps[(r_b)](t_N) \\
    \end{array}\right]. \label{eq:2:d}
\end{align} \end{subequations}
下面展示了一组带有左大括号和子公式编号的公式~\ref{eq:3}:
\begin{subequations} \label{eq:3} \begin{empheq}[left={\empheqlbrace\,}]{align}
    \vec{x}(n+1) &= \mat{A}\sps[d] \vec{x}(n) + \mat{B}\sps[d] \vec{u}(n) + \vec{z}\sps[x](n), \label{eq:3:a} \\
    \vec{y}\sps[n](n) &= \mat{C}\sps[d] \vec{x}(n) + \mat{D}\sps[d] \vec{u}(n) + \vec{z}\sps[y](n), \label{eq:3:b} \\
    \vec{u}\sps[n](n) &= \vec{u}(n) + \vec{z}\sps[u](n). \label{eq:3:c}
\end{empheq} \end{subequations}

\section{引用排版示例}
\label{sec:intro:reference}
% 引用
我是一段带有引用的文字~\cite{IEEE1363}。
% 句内引用
我是一段在一句话内引用了文献~\parencite{Jeyakumar2004} 的文字,需要使用 \verb|parencite| 命令。
% 句间引用
我是一段在一句话的结尾带有引用的文字~\cite{ElIdrissi1994},需要使用 \verb|cite| 命令~\cite{You2024}。
% 参考文献位置
请注意要将参考文献放在 \verb|data/ref.bib| 文件中,建议使用 Zotero 软件,配合 Better Bib(La)TeX 插件完成这一工作。

\section{图片排版示例}
\label{sec:intro:figure}
% 图片
% 单个图片
如图~\ref{fig:1} 展示了如何插入单张图片。
\begin{figure}[ht] % h (here), t (top), b (bottom), p (page), ! (mandatory in here?)
    \centering \includegraphics[width=.5\textwidth]{figure_1.png}
    \caption{我是一张图片} \label{fig:1}
\end{figure}
% 子图
如图~\ref{fig:2} 展示了如何插入含多个子图的图片,位图可以使用 jpg, png 等格式。
\begin{figure}[ht] \centering
    \subcaptionbox{子图 1 \label{fig:2:a}}  {\includegraphics[height=.2\textwidth]{figure_2_a.png}} \hspace{2em} % 水平间距
    \subcaptionbox{子图 2 \label{fig:2:b}}  {\includegraphics[height=.2\textwidth]{figure_2_b.png}} \\ \vspace{1em} % 垂直间距
    \subcaptionbox{子图 3 \label{fig:2:c}}  {\includegraphics[height=.2\textwidth]{figure_2_c.png}} \hspace{2em} % 水平间距
    \subcaptionbox{子图 4 \label{fig:2:d}}  {\includegraphics[height=.2\textwidth]{figure_2_d.png}}
    \caption{我是一组图片} \label{fig:2}
\end{figure}
% 矢量图
矢量图建议转换为 pdf 格式,如图~\ref{fig:3} 所示。
\begin{figure}[ht] \centering
    \includegraphics[width=.5\textwidth]{figure_3.pdf}
    \caption{我是一张矢量图片} \label{fig:3}
\end{figure}
% 也可以使用 tikz 制图
同样可以使用 \verb|tikz| 包制作矢量图,如图~\ref{fig:4} 所示。
\begin{figure}[htbp] \centering 
    \begin{tikzpicture}[node distance = 8pt]
        % nodes
        % frame 1 - measurement
        \node[roundrect node]
            (timeinit) { 生成满足定义~\ref{eq:3} 的采样时刻~$\mathbb{T}\sps[ua]$, $\mathbb{T}\sps[ya]$ };
        \node[roundrect node, below = of timeinit]
            (experiment) { 根据采样时刻进行实验,收集数据~$\mathbb{S}\sps[ua]$、$\mathbb{S}\sps[ya]$ };
        % frame 2 - calculation
        \node[roundrect node, below = of experiment, yshift = -16pt]
        (freqest) { 计算不变子空间参数~$\hat{\vec{U}}\sps[a]$、$\hat{\vec{Y}}\sps[a]$ };
        \node[roundrect node, below = of freqest]
            (sysest) { 计算系统参数~$\hat{\vec{\theta}}_0$ };
        % output
        % \node[roundrect node, below = of sysest, yshift = -2pt]
        %     (algoend) {Output $\hat{G}(p)$};
        
        % arrowed lines
        \path[arrows={->[scale=1.2]}, thick]
            (timeinit) edge (experiment)
            (experiment) edge (freqest)
            (freqest) edge (sysest);
        %     (sysest) edge (algoend);

        % frames
        \begin{scope}[on background layer]
            \node[
                fit=(timeinit) (experiment),
                label={[gray, anchor = south west]north west: 测量过程}] {};
            \node[
                fit=(freqest) (sysest),
                label={[gray, anchor = south west]north west: 计算过程}] {};
        \end{scope}
    \end{tikzpicture}
    \caption{我是一张 tikz 制作的图片} \label{fig:4}
\end{figure}

\section{表格排版示例}
\label{sec:intro:table}
表格可以使用 excel2latex 插件导出,可以标注附注,如表~\ref{table:discrete:complexity} 所示。
\begin{table}[htb]
    \centering \caption{我是一张带有附注的表格} \label{table:discrete:complexity}
    \begin{threeparttable} \begin{tabularx}{0.8\textwidth}{
            |>{\hsize=\hsize \linewidth=\hsize \centering\arraybackslash}X % 居中对齐
            |>{\hsize=\hsize \linewidth=\hsize \centering\arraybackslash}X % 居中对齐
            |>{\hsize=\hsize \linewidth=\hsize \raggedright\arraybackslash}X % 靠左对齐
        |}
        \hline \textbf{辨识任务} & \textbf{辨识方法} & \makecell{\centering \textbf{计算复杂度}} \\
        \hline \multirow{5}{*}{仅估计系统参数}
            & 算法1 & $\mathcal{O}(Nq^2 + q^2\bar{n} + \bar{n}^3)$ \\
        \cline{2-3} & \multirow{2}{*}{算法2} & \multirow{2}{*}{\makecell[l]{
            $\mathcal{O}(Nq + q^2\bar{n} + \bar{n}^3)$ or \\
            $\mathcal{O}(N\log T + q^2\bar{n} + \bar{n}^3)$}} \\
            & & \\
        \cline{2-3} & 算法3\tnote{1} & $\mathcal{O}(N\bar{n}^2 + \bar{n}^3)$ \\
        \cline{2-3} & 算法4\tnote{2} & $\mathcal{O}(N\log T + q^2\bar{n} + \bar{n}^3)$ \\
        \hline \multirow{2}{*}{\makecell{同时估计系统参数和 \\ 扰动参数}}
            & 算法2 & $\mathcal{O}(N\log N + q^2\bar{n} + \bar{n}^3)$ \\
        \cline{2-3} & 算法3 & $\mathcal{O}(N\bar{n}^2 + \bar{n}^3)$ \\
        \hline
    \end{tabularx}
    \begin{tablenotes}
        \small
        \item{1} 参见文献~\cite{Subspace_Overall_Overschee1996};
        \item{2} 参见文献~\cite{Subspace_Freq_McKelvey1996}。
    \end{tablenotes}
    \end{threeparttable}
\end{table}

\section{算法排版示例}
\label{sec:intro:algorithm}
本模板使用的是 algorithm2e 包,如算法~\ref{algorithm:1} 所示。
\begin{algorithm}[htb]
    \LinesNumbered \SetAlgoLined
    \caption{我是一个带有注释的算法} \label{algorithm:1}
    \KwData{$\mat{Y}_N\sps[nd]$、$\mat{U}_N\sps[nd]$。}
    \KwResult{$\hat{\mat{\Xi}}\ups[yy,l]\sps[d]$、$\hat{\mat{\Xi}}\ups[uu]\sps[d]$。}
    
    \tcc{1. 计算参数1}
    $\hat{\mathbcal{Y}}\sps[rd] \leftarrow \FFT(\mat{Y}_N\sps[nd])$、$\hat{\mathbcal{U}}\sps[rd] \leftarrow \FFT(\mat{U}_N\sps[nd])$ \tcpright{逐列计算 FFT}
    $\hat{\mathbcal{Y}}_{\text{切片:}[0:T:(k-1)T,:]}\sps[rd] \leftarrow 0$、$\hat{\mathbcal{U}}_{\text{切片:}[0:T:(k-1)T,:]}\sps[rd] \leftarrow 0$\;

    \tcc{2. 计算参数2}
    $\hat{\mathbcal{Y}}\ups[1]\sps[rd] \leftarrow \RESHAPE(\hat{\mathbcal{Y}}\sps[rd],[p \; 1 \; N])$、$\hat{\mathbcal{Y}}\ups[2]\sps[rd] \leftarrow \overline{\RESHAPE(\hat{\mathbcal{Y}}\sps[rd],[1 \; p \; N])}$\;
    $\hat{\mathbcal{Y}}\ups[yy]\sps[rd] \leftarrow \PAGEMTIMES(\hat{\mathbcal{Y}}\ups[1]\sps[rd], \hat{\mathbcal{Y}}\ups[2]\sps[rd])$\;
    $\hat{\mathbcal{y}}\ups[yy]\sps[rd] \leftarrow \IFFT(\hat{\mathbcal{Y}}\ups[yy]\sps[rd])$ \tcpright{对最后一维计算 IFFT}
    $\hat{\mat{\Xi}}\ups[yy,l]\sps[d] \leftarrow \frac{1}{N} \times \RESHAPE[(\hat{\mathbcal{y}}\ups[yy]\sps[rd])_{\text{切片:}[:,:,l]}] - \sum_{n=0}^{l-1}{x(n)\overline{y(N-l+n)}}$\;

    \tcc{3. 计算参数3}
    $\hat{\mat{\Xi}}\ups[uu]\sps[d] \leftarrow \frac{1}{N^2} \times (\hat{\mathbcal{U}}\sps[rd])\tr \overline{\hat{\mathbcal{U}}\sps[rd]}$\;

    \Return $\hat{\mat{\Xi}}\ups[yy,l]\sps[d]$($l = 0,\ldots,M-1$)、$\hat{\mat{\Xi}}\ups[uu]\sps[d]$.
\end{algorithm}
